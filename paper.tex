\documentclass[11pt]{sty/oecu-thesis}
\usepackage{cite}
\usepackage[dvipdfmx]{graphicx}
\usepackage{sty/subfigure}
\usepackage{subcaption}
\usepackage{sty/lcaption}
\usepackage{times}
\usepackage{url}
\usepackage{amsmath}

% もう少し具体的なタイトルにする。
\title[タイトルタイトル]{タイトル\\タイトル}
\author{名前 名前}
\date{{平成}\rensuji{99}年\rensuji{99}月\rensuji{99}日}
\学生番号{HT99A999}
\指導教員{久松 潤之 准教授}

% 特別研究の場合はコメントをはずす.卒業研究の場合はコメントアウトする.
%\論文種別{特別研究論文}
\年度{平成99}

\所属{総合情報学部 情報学科}


\begin{document}


\newcommand{\insertfigeps}[2]{%
  \begin{figure}[tb]
    \begin{center}
      \leavevmode
      \includegraphics[width=.80\textwidth]%
      {figure/#1.eps}
      \lcaption{#2}
      \label{fig:#1}
    \end{center}
  \end{figure}}

\newcommand{\insertfigpng}[2]{%
  \begin{figure}[tb]
    \begin{center}
      \leavevmode
      \includegraphics[width=.80\textwidth]%
      {figure/#1.png}
      \lcaption{#2}
      \label{fig:#1}
    \end{center}
  \end{figure}}



%\makeextratitle
\maketitle
\pagenumbering{roman}
\begin{abstract}
  アブストアブストアブストアブストアブストアブストアブストアブストアブスト
  
\keywords % 主な用語
  キーワード1\quad
  キーワード2\quad
  キーワード3\quad
\end{abstract}


\tableofcontents
% 以下の二つは,論文のフォーマットにそっていないが,
% 確認用のためにつける.論文提出時には,コメントアウトする.
\listoffigures
\listoftables
\cleardoublepage

\setcounter{page}{1}
\pagenumbering{arabic}

\chapter{はじめに}
\label{cha:intro}

ここにイントロを書く

本論文の構成は以下の通りである.
まず,\ref{cha:related} 章では,~~.
\ref{cha:xxx} 章では,~~.
最後に,\ref{cha:conclusion} 章では,本論文のまとめと今後の課題を述べる.
\chapter{関連研究}
\label{cha:related}
引用の書き方

例えば、\cite{NS2} では、~~

\chapter{XXX}
\label{cha:xxx}
このファイルは、ファイル名や章タイトル、そして、label を適宜書き換えること。

\insertfigeps{sample-eps}{eps 画像の貼り付けの例}
\insertfigpng{sample-png}{png 画像の貼り付けの例}
\chapter{まとめと今後の課題}
\label{cha:conclusion}

 本稿では,~~.
 
 今後の課題としては、~~.
\acknowledgment
本研究と本論文を終えるにあたり、御指導、御教授を頂いた久松潤之准教授に
深く感謝致します。また、学生生活を通じて、基礎的な学問、学問に取り組む
姿勢を御教授頂いた、登尾啓史教授、升谷保博教
授、渡邊郁教授、南角茂樹教授、鴻巣敏之教授、北嶋暁教授、大西克彦
准教授、小枝正直准教授に深く感謝致します。

本研究期間中、本研究に対する貴重な御意見、御協力を頂きました久松研究室
の皆様に心から御礼申し上げます。


\bibliographystyle{junsrt}
\bibliography{bib/myrefs}


\end{document}
