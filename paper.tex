\documentclass[11pt]{sty/oecu-thesis}
\usepackage{cite}
\usepackage[dvipdfmx]{graphicx}
\usepackage{sty/subfigure}
\usepackage{subcaption}
\usepackage{sty/lcaption}
\usepackage{times}
\usepackage{url}
\usepackage{amsmath}

% もう少し具体的なタイトルにする。
\title[タイトルタイトル]{タイトル\\タイトル}
\author{名前 名前}
\date{{平成}\rensuji{99}年\rensuji{99}月\rensuji{99}日}
\学生番号{HT99A999}
\指導教員{久松 潤之 准教授}

% 特別研究の場合はコメントをはずす.卒業研究の場合はコメントアウトする.
%\論文種別{特別研究論文}
\年度{平成99}

\所属{総合情報学部 情報学科}


\begin{document}

\input{tex/macro}

%\makeextratitle
\maketitle
\pagenumbering{roman}
\begin{abstract}
  アブストアブストアブストアブストアブストアブストアブストアブストアブスト
  
\keywords % 主な用語
  キーワード1\quad
  キーワード2\quad
  キーワード3\quad
\end{abstract}


\tableofcontents
% 以下の二つは,論文のフォーマットにそっていないが,
% 確認用のためにつける.論文提出時には,コメントアウトする.
\listoffigures
\listoftables
\cleardoublepage

\setcounter{page}{1}
\pagenumbering{arabic}

\chapter{はじめに}
\label{cha:intro}

ここにイントロを書く

本論文の構成は以下の通りである.
まず,\ref{cha:related} 章では,~~.
\ref{cha:xxx} 章では,~~.
最後に,\ref{cha:conclusion} 章では,本論文のまとめと今後の課題を述べる.
\chapter{関連研究}
\label{cha:related}
引用の書き方

例えば、\cite{NS2} では、~~

\input{tex/xxx}
\input{tex/conclusion}
\input{tex/acknowledgements}

\bibliographystyle{junsrt}
\bibliography{bib/myrefs}


\end{document}
